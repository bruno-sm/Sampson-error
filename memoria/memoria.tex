%%%%%%%%%%%%%%%%%%%%%%%%%%%%%%%%%%%%%%%%%
% Short Sectioned Assignment LaTeX Template Version 1.0 (5/5/12)
% This template has been downloaded from: http://www.LaTeXTemplates.com
% Original author:  Frits Wenneker (http://www.howtotex.com)
% License: CC BY-NC-SA 3.0 (http://creativecommons.org/licenses/by-nc-sa/3.0/)
%%%%%%%%%%%%%%%%%%%%%%%%%%%%%%%%%%%%%%%%%

%----------------------------------------------------------------------------------------
%	PACKAGES AND OTHER DOCUMENT CONFIGURATIONS
%----------------------------------------------------------------------------------------

\documentclass[paper=a4, fontsize=11pt]{scrartcl} % A4 paper and 11pt font size

% ---- Entrada y salida de texto -----

\usepackage[T1]{fontenc} % Use 8-bit encoding that has 256 glyphs
\usepackage[utf8]{inputenc}
%\usepackage{fourier} % Use the Adobe Utopia font for the document - comment this line to return to the LaTeX default

% ---- Idioma --------

\usepackage[spanish, es-tabla]{babel} % Selecciona el español para palabras introducidas automáticamente, p.ej. "septiembre" en la fecha y especifica que se use la palabra Tabla en vez de Cuadro

% ---- Otros paquetes ----

\usepackage{url} % ,href} %para incluir URLs e hipervínculos dentro del texto (aunque hay que instalar href)
\usepackage{amsmath,amsfonts,amsthm} % Math packages
%\usepackage{graphics,graphicx, floatrow} %para incluir imágenes y notas en las imágenes
\usepackage{graphics,graphicx, float} %para incluir imágenes y colocarlas
\usepackage{listings}
\usepackage{color}
\usepackage{algorithm2e}
\usepackage{csvsimple}
\usepackage{pgfplots}
\pgfplotsset{compat=1.14}
\usepackage{pgf}
\usepackage{gnuplot-lua-tikz}
\usepackage{tikz}

% Para hacer tablas comlejas
%\usepackage{multirow}
%\usepackage{threeparttable}

%\usepackage{sectsty} % Allows customizing section commands
%\allsectionsfont{\centering \normalfont\scshape} % Make all sections centered, the default font and small caps

\usepackage{fancyhdr} % Custom headers and footers
\pagestyle{fancyplain} % Makes all pages in the document conform to the custom headers and footers
\fancyhead{} % No page header - if you want one, create it in the same way as the footers below
\fancyfoot[L]{} % Empty left footer
\fancyfoot[C]{} % Empty center footer
\fancyfoot[R]{\thepage} % Page numbering for right footer
\renewcommand{\headrulewidth}{0pt} % Remove header underlines
\renewcommand{\footrulewidth}{0pt} % Remove footer underlines
\setlength{\headheight}{13.6pt} % Customize the height of the header

\numberwithin{equation}{section} % Number equations within sections (i.e. 1.1, 1.2, 2.1, 2.2 instead of 1, 2, 3, 4)
\numberwithin{figure}{section} % Number figures within sections (i.e. 1.1, 1.2, 2.1, 2.2 instead of 1, 2, 3, 4)
\numberwithin{table}{section} % Number tables within sections (i.e. 1.1, 1.2, 2.1, 2.2 instead of 1, 2, 3, 4)

\setlength\parindent{0pt} % Removes all indentation from paragraphs - comment this line for an assignment with lots of text

\newcommand{\horrule}[1]{\rule{\linewidth}{#1}} % Create horizontal rule command with 1 argument of height

\definecolor{mygreen}{rgb}{0,0.6,0}
\definecolor{mygray}{rgb}{0.5,0.5,0.5}
\definecolor{mymauve}{rgb}{0.58,0,0.82}
\definecolor{backgray}{rgb}{0.9,0.9,0.9}
\definecolor{fontgray}{rgb}{0.2,0.2,0.2}

\lstdefinestyle{command}{ %
  backgroundcolor=\color{backgray},   % choose the background color; you must add \usepackage{color} or \usepackage{xcolor}; should come as last argument
  language=Bash,
  basicstyle=\scriptsize\ttfamily,        % the size of the fonts that are used for the code
  breakatwhitespace=false,         % sets if automatic breaks should only happen at whitespace
  breaklines=true,                 % sets automatic line breaking
  captionpos=b,                    % sets the caption-position to bottom
  commentstyle=\color{mygreen},    % comment style
  escapeinside={\%*}{*)},          % if you want to add LaTeX within your code
  extendedchars=true,              % lets you use non-ASCII characters; for 8-bits encodings only, does not work with UTF-8
  keepspaces=true,                 % keeps spaces in text, useful for keeping indentation of code (possibly needs columns=flexible)
  keywordstyle=\color{blue},       % keyword style
  numbers=none,                    % where to put the line-numbers; possible values are (none, left, right)
  showspaces=false,                % show spaces everywhere adding particular underscores; it overrides 'showstringspaces'
  showstringspaces=false,          % underline spaces within strings only
  showtabs=false,                  % show tabs within strings adding particular underscores
  stringstyle=\color{mymauve},     % string literal style
  tabsize=2,	                   % sets default tabsize to 2 spaces
  title=\lstname,                   % show the filename of files included with \lstinputlisting; also try caption instead of title
  framexleftmargin=4pt,
  framextopmargin=1pt,
  framexbottommargin=1pt, 
  frame=tb,
  framerule=0pt
}

\lstdefinestyle{code}{
  backgroundcolor=\color{backgray},   % choose the background color; you must add \usepackage{color} or \usepackage{xcolor}; should come as last argument
  basicstyle=\scriptsize\ttfamily,        % the size of the fonts that are used for the code
  breakatwhitespace=false,         % sets if automatic breaks should only happen at whitespace
  breaklines=true,                 % sets automatic line breaking
  captionpos=b,                    % sets the caption-position to bottom
  commentstyle=\color{mygreen},    % comment style
  escapeinside={\%*}{*)},          % if you want to add LaTeX within your code
  extendedchars=true,              % lets you use non-ASCII characters; for 8-bits encodings only, does not work with UTF-8
  keepspaces=true,                 % keeps spaces in text, useful for keeping indentation of code (possibly needs columns=flexible)
  keywordstyle=\color{blue},       % keyword style
  showspaces=false,                % show spaces everywhere adding particular underscores; it overrides 'showstringspaces'
  showstringspaces=false,          % underline spaces within strings only
  showtabs=false,                  % show tabs within strings adding particular underscores
  stringstyle=\color{mymauve},     % string literal style
  tabsize=2,	                   % sets default tabsize to 2 spaces
  title=\lstname,                  % show the filename of files included with \lstinputlisting; also try caption instead of 
  numbers=left,
  numbersep=5pt,
  numberstyle=\tiny\color{mygray},
  framexleftmargin=4pt,
  framextopmargin=1pt,
  framexbottommargin=1pt, 
  frame=tb,
  framerule=0pt
}



\title{	
\normalfont \normalsize 
\textsc{\textbf{Visión por computador (2018-2019)} \\ Doble Grado en Informática y Matemáticas \\ Universidad de Granada} \\ [25pt]
\horrule{0.5pt} \\[0.4cm]
\huge Estimación de homografías usando el error de Sampson \\
\horrule{1pt} \\[0.5cm]
}
\author{Iñaki Madinabeitia Cabrera\\
Bruno Santidrián Manzanedo}
\date{\normalsize\today}



\begin{document}
\maketitle
\newpage
\tableofcontents
\newpage

\section{Intoducción}
Este trabajo consiste en implementar un algoritmo de estimación de homografías que utilice el error de Samson, tal como se describe en \cite{mvg}, como medida de su bondad. Esto es que, dados dos conjuntos de puntos, el algoritmo debe aproximar una homografía que lleve un conjunto en el otro cuyo error de Sampson sea razonablemente bajo.\\

Para el proceso de aproximación de la homografía \cite{mvg} recomienda el uso de algoritmos iterativos, los cuales, a grandes rasgos, se componen de los siguientes elementos:\\

\begin{itemize}
  \item \textbf{Función de coste:} En nuestro caso será el error de Sampson.\\
  
  \item \textbf{Parametrización:} Dado que el objeto a minimizar es una homografía los parámetros serán simplemente los elementos que conforman la matriz.\\
  
  \item \textbf{Especificación de la función:} Se debe especificar una función que exprese el coste en términos de un conjunto de parámetros. En nuestro caso calculará el error de Sampson a partir de una homografía y dos conjuntos de puntos; la implementación de esta función se puede encontrar en \textit{src/sampson.py}.\\
  
  \item \textbf{Inicialización:} Se calcula una homografía inicial. Para ello nosotros hemos decidido utilizar RANSAC, ya que aunque es más costoso que DLT, tener una solución inicial menos afectada por los outliers mejora el resultado final. La implementación de RANSAC se puede encontrar en \textit{src/ransac.py}.\\
  
  \item \textbf{Iteración:} IÑAKI.\\
\end{itemize}

Podemos, por tanto, distinguir tres elementos importantes: El error de Sampson, \mbox{RANSAC} y el algoritmo iterativo. Pasemos a hablar de cada uno de forma más detallada.


\section{Detalles y decisiones de implementación}

\subsection{Error de Sampson}
Para el cálculo del error de sampson hemos usado la fórmula (4.13) descrita en la página 100 de \cite{mvg}, es decir
$$D_{\bot} = \sum_i{\epsilon_i^T (J_i J_i^T)^{-1} \epsilon_i}$$

Donde si $H$ es la matriz de la homografía, $h$ son los elementos de esta en forma de columna, $A_i$ es la matriz construida en DLT y $x_i = (x_i, y_i, 1)^T$ y $x_i' = (x_i', y_i', 1)^T$ son dos puntos en correspondencia, entonces:

$$\epsilon_i = A_ih$$

$$J_i = \frac{\partial A_ih}{\partial (x_i, y_i, x_i', y_i')}$$

Tras efectuar los cálculos tenemos:

$$J_i =
  \begin{bmatrix}
    h_7 y_i' - h_4 & h_8 y_i' - h_5 & 0 &  h_7 x_i + h_8 y_i + h_9\\
    h_1 - h_7 x_i' & h_2 - h_8 x_i' & -h_7 x_i - h_8 y_i - h_9 & 0\\
  \end{bmatrix}
$$

Donde $h_1,...,h_9$ son los elementos de $h$.\\

Esta fórmula se implementa en la función \textit{sampson\_error(pts1, pts2, H)} del archivo \textit{src/sampson.py}.


\subsection{RANSAC}

Para la implementación de RANSAC nos hemos basado en el apartado 4.7.1 de \cite{mvg}. Le damos los siguientes valores a los parámetros:

\begin{itemize}
  \item \textbf{Probabilidad inicial de que un punto sea un outlier (v):}
  Para estimar la probabilidad de que un punto sea un outlier usamos la aproximación adaptativa que se describe en la página 120, esto nos permite aproximarnos a la proporción real sin tener que analizar experimentalmente cada conjunto de correspondencias. Inicialmente se le otroga un valor de $0.6$.\\
  
  \item \textbf{Número de iteraciones (N):}
  Se calcula dinámicamente a medida que se acualiza $v$, para ello se utliliza la fórmula (4.18):
  $N = \frac{log(1-p)}{log(1-(1-v)^4)}$ donde $p=0.99$ es la proporción de inliers que esperamos obtener.\\
  
  \item \textbf{Tolerancia:}
  Consideramos que un punto es un outlier si presenta un error de más de 3 píxeles.\\
  
  \item \textbf{Tamaño aceptable del conjunto de inliers (T):}
  También se calcula dinámicamente a medida que se acualiza $v$, para ello se utliliza la fórmula presentada en la página 120: $T = (1-v)n$ donde $n$ es el número de correspondencias.\\
\end{itemize}

RANSAC utiliza DLT para calcular las homografías, este está implementado tal y como se indica en los apartados 4.1 y 4.1.1, usando la ecuación simplificada (4.3) y permitiendo soluciones sobre-determinadas. De esta forma RANSAC puede calcular la homografía final usando todos los inliers que ha detectado, lo que le dota de una robustez extra.\\

Tanto RANSAC como DLT están implementados en el archivo \textit{src/ransac.py}.

\subsection{Algoritmo iterativo}
IÑAKI

\section{Resultados y valoraciones}

\section{Posibles mejoras}

\newpage
\nocite{*}
\bibliographystyle{plain}
\bibliography{citas}
\end{document}
